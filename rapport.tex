\documentclass{article}
\usepackage{graphicx} % Required for inserting images
\usepackage[utf8]{inputenc}
\usepackage{amsthm}
\usepackage{amsmath}
\usepackage{amssymb}
\usepackage{mathtools}
\usepackage{hyperref}
\usepackage[a4paper]{geometry}

\title{Caractérisation des singularités de type $\mathfrak{J}$}
\author{Félix Larose-Gervais}
\date{Mai 2023}


\newtheorem{definition}{Définition}
\newtheorem{proposition}{Proposition}
\newtheorem{conjecture}{Conjecture}
\newtheorem{example}{Exemple}

\begin{document}

\maketitle

\newpage

\tableofcontents

\newpage

\section{Introduction}

\subsection{Notations}

Soit $a, b \in \mathbb{Z}$ on note la relation de coprimalité $\perp$

\[ a \perp b \iff \gcd(a, b) = 1 \]
Soit $m, n \in \mathbb{N}$, $X$ un ensemble, notons 

\begin{itemize}
    \item $Sym(X)$ le groupe de bijections de $X$ dans lui-même
    \item $Sym(m)$ le groupe de bijections $Sym(\{1, \dots, m \})$ 
    \item $\mathbb{Z}_n$ l'anneau des entiers modulo n ($\mathbb{Z}/n\mathbb{Z}$)
    \item $\mathbb{Z}_n^\times$ son groupe d'inversibles ($\{ a \in \mathbb{Z}_n \mid a \perp n \})$
    \item $X^m$ les m-uplets de $X$ $(\underbrace{X \times \cdots \times X}_{m fois})$
\end{itemize}
On notera aussi $S_n^m$ les m-uplets d'inversibles modulo n $({\mathbb{Z}_n^\times}^m)$

\subsection{Définitions}

\begin{definition}
    Une \textbf{singularité} est un $[a] = ([a_1], \dots, [a_m]) \in S_n^m$, on appelle \begin{itemize}
        \item $n$ la \textbf{racine} de la singularité
        \item $[a_1], \dots, [a_m]$ les \textbf{poids} de la singularité
    \end{itemize}
\end{definition}

\begin{definition}
    Un \textbf{éclatement} $a \in \mathbb{Z}^m$ d'une singularité $[a] \in S_n^m$ (noté $a \in [a]$) est un choix de 
    représentant $a = (a_1, \dots, a_m)$ tel que
    \begin{align*}
        \forall i \neq j & : a_i \perp a_j
    \end{align*}
    On note $E_a$ l'ensemble des singularités associées à l'éclatement $a$ comme suit:
    \begin{align*}
        E_a & = \{ [a^i] \in S_{a_i}^m \mid \forall i = 1..m,\; a_i > 1 \} \\
        [a^i] & = ([a^i_1], \dots, [a^i_m]) \\
        [a^i_j] & \equiv \begin{cases}
            -n & \text{si $i = j$} \\
            a_j & \text{sinon}
        \end{cases} \pmod{a_i} \quad \forall j = 1..m
    \end{align*}
    On appelle $a = (a_1, \dots, a_m)$ \textbf{l'éclatement naturel} de $[a]$ si 
    les $a_1, \dots, a_m$ sont les plus petits représentant positifs de leurs classes
\end{definition}

\begin{definition}
    Un éclatement $a \in [a]$ est dit \textbf{lisse} si $a = (1, \dots, 1)$
\end{definition}

\begin{definition}
    La singularité $[a]$ est dite de \textbf{type $\mathfrak{J}$} (noté $[a] \in \mathfrak{J}$) ssi
    \begin{align*}
        \exists a \in [a] : \forall [a^i] \in E_a : [a^i] \in \mathfrak{J}
    \end{align*}
\end{definition}

\newpage

\subsection{Rappels d'arithmétique}

\subsubsection{Algorithme d'Euclide et PGCD}

Soit $a, b \in \mathbb{Z}$, on calcule le PGCD comme suit
\begin{align*}
    \gcd(a, b) & := \begin{cases}
        a & \text{ si $b = 0$} \\
        \gcd(b, a \mod b) & \text{ sinon}
    \end{cases}
\end{align*}
Avec $k \in \mathbb{Z}$, on a les propriétés suivantes:
\begin{align*}
    \gcd(a, 1) & = 1 \\
    \gcd(a, b) & = \gcd(b, a) \\
    \gcd(a, b) &= \gcd(a, -b) \\
    \gcd(a, b) & = \gcd(a, b + ka)
\end{align*}
De la dernière on déduit directement, pour $n \in \mathbb{N}$
\begin{align*}
        a \equiv b \pmod n \implies \gcd(a, n) = \gcd(b, n)
\end{align*}

\subsubsection{Théorème des restes chinois}

Soit $m, n_1, \dots, n_m \in \mathbb{N}$ et $a_1, \dots, a_m \in \mathbb{Z}$, notons le produit $n = n_1 \cdots n_m$

Si $\forall i \neq j : n_i \perp n_j$, alors $\exists!x \in \mathbb{Z}_n$ tel que
\begin{align*}
    x &\equiv a_1 \pmod {n_1} \\
    &\vdotswithin{\equiv} \\
    x &\equiv a_m \pmod {n_m}
\end{align*}


\subsection{Résultats connus}
Résultats utiles, dûs à Habib Jaber.

\begin{proposition}
    Soit $a_1, a_2 \in \mathbb{Z},\; \gcd(a_1, a_2) = 1$, alors
    \[ {[(a_1, a_2)]}_{a_1+a_2} \in \mathfrak{J} \]
\end{proposition}

\begin{example}
    $\gcd(2, 1) = 1 \implies {[(2, 1)]}_3 \in \mathfrak{J}$
\end{example}

\begin{proposition}
    \[ 
    {[(a_1, a_2)]}_n \in \mathfrak{J} \iff 
    \forall k \in \mathbb{Z}: {[(a_1, a_2)]}_{n+ka_1a_2} \in \mathfrak{J} 
    \]
\end{proposition}

\begin{example}
    ${[(2, 1)]}_3 \in \mathfrak{J} \implies {[(2, 1)]}_5 \in \mathfrak{J},\; {[(2, 1)]}_7 \in \mathfrak{J}, \dots$
\end{example}

\newpage

\section{Propositions}

\subsection{Ordre des poids}

Soit $\sigma \in Sym(m)$, et ces permutations associées $\pi_{\sigma} \in Sym(S_n^m)$

\begin{proposition}
    L'ordre des poids d'une singularité n'affecte pas le type $\mathfrak{J}$
    \[ [a] \in \mathfrak{J} \implies \pi_{\sigma}([a]) \in \mathfrak{J} \]
\end{proposition}

\begin{proof}
    Pour $a = (a_1, \dots, a_m) \in [a]$ tel que $\forall [a^i] \in E_a : [a^i] \in \mathfrak{J}$, on a

    \begin{enumerate}
    \item Cas de base: $E_a = \emptyset$

        On a donc $a = (1, \dots, 1)$

        Ainsi $[a] = ([1], \dots, [1]) = \pi_{\sigma}([a]) \in \mathfrak{J}$

    \item Induction structurelle

        Supposons $\forall [a^i] \in E_a : [a^i] \in \mathfrak{J} \implies \pi_{\sigma}([a^i]) \in \mathfrak{J}$
        \begin{align*}
            E_a & = \{ [a^i] \in S_{a_i}^m \mid \forall i = 1..m,\; a_i > 1 \} \\
            [a^i] & = ([a_1^i], \dots, [a_m^i]) \\
            [a_j^i] &\equiv \begin{cases}
                    -n &\text{si } i = j \\
                    a_j &\text{sinon}
                \end{cases} \pmod{a_i} \quad \forall j = 1..m
        \end{align*}

        Considérons, pour $b = (b_1, \dots, b_m) = (a_{\sigma(1)}, \dots, a_{\sigma(m)}) \in \pi_{\sigma}([a])$
        \begin{align*}
            E_b & = \{ [b^i] \in S_{b_i}^m \mid \forall i = 1..m,\; b_i > 1 \} \\
            [b^i] & = ([b_1^i], \dots, [b_m^i]) \\
            [b_j^i] & \equiv \begin{cases}
                    -n &\text{si } i = j \\
                    b_j &\text{sinon}
                \end{cases} \pmod{b_i} \quad \forall j = 1..m \\
                & \equiv \begin{cases}
                    -n &\text{si } \sigma(i) = \sigma(j) \\
                    a_{\sigma(j)} &\text{sinon}
                \end{cases} \pmod {a_{\sigma(i)}} \\
                & \equiv [a_{\sigma(j)}^{\sigma(i)}] \\
            [b^i] & = ([a_{\sigma(1)}^{\sigma(i)}], \dots, [a_{\sigma(m)}^{\sigma(i)}]) \\
                & = \pi_{\sigma}([a^{\sigma(i)}]) \in \mathfrak{J}
        \end{align*}

        Ainsi $\pi_{\sigma}([a]) \in \mathfrak{J}$
    \end{enumerate}
\end{proof}

\begin{example}
    Sachant ${[(3, 2)]}_5 \in \mathfrak{J}$, on en déduit ${[(2, 3)]}_5 \in \mathfrak{J}$
\end{example}

\newpage

\begin{proposition}
    (strict, rework) Soit $a=(a_0, a_1, a_2)$, alors
    \[ a \in \mathfrak{J} \implies a_0 \geq a_1 + a_2 \]
\end{proposition}

\begin{proof}
    Supposons $a_0 < a_1 + a_2$

    Si $a_1 = a_2$, alors $\neg \mathfrak{J}(a)$

    Sinon, $a_1 \neq a_2$, supposons sans perdre de généralité que $a_1 > a_2$

    Considérons l'éclatement $a^1 = (a_1,\; -a_0 \mod a_1,\; a_2 \mod a_1) \in E_a$
    \begin{align*}
        a_1 > a_2 & \implies 2a_1 > a_1+a_2               \\
                  & \implies (-a_0 \mod a_1) = 2a_1 - a_0 \\
        a_1 > a_2 & \implies (a_2 \mod a_1) = a_2
    \end{align*}

    On a donc $a^1 = (a_1, 2a_1-a_0, a_2)$

    Puisque $a_0 < a_1 + a_2$, on a $a_1 < 2a_1 - a_0 + a_2$

    Donc $a^1$ vérifie la condition initiale, on répète le raisonnement avec $a^1$
\end{proof}

\begin{example}
    $(5, 4, 3) \not \in \mathfrak{J}$ car $5 < 7$
\end{example}

\subsection{Existence d'un éclatement}

\begin{proposition}
    Toute singularité admet un éclatement
\end{proposition}

\begin{proof}
    Soit $[a] = ([a_1], \dots, [a_m]) \in S_n^m$

    Prenons $(a_1, \dots, a_m) \in [a]$ son représentant naturel

    On cherche $(b_1, \dots, b_m) \in [a]$ tels que $\forall i \neq j : b_i \perp b_j$ et $\forall i : b_i \perp n$

    Il suffit de prendre $b_1 = a_1$ et $\forall i = 2..m$, un $b_i$ vérifiant
    \begin{align*}
        b_i & \equiv a_i \pmod n \\
        b_i & \equiv 1 \pmod {b_1} \\
            & \vdotswithin{\equiv} \\
        b_i & \equiv 1 \pmod {b_{i-1}} 
    \end{align*}

    De tels $b_i$ existent par le théorème des restes chinois

    On vérifie les coprimalités nécéssaires grâce aux propriétés de $\gcd$

    On a par la première congruence $\forall i : b_i \perp n$ (puisque $\forall i : a_i \perp n$)

    Et par les suivantes $\forall i \neq j : b_i \perp b_j$

    On a donc $b = (b_1, \dots, b_m)$ un éclatement de $[a]$
\end{proof}

\newpage

\subsection{Caractérisation des singularités de type \texorpdfstring{$\mathfrak{J}$}{J} à 1 poids}

On sait déjà que la condition de coprimalité est nécéssaire à être de type $\mathfrak{J}$. 

On montre maintenant qu'elle est en fait suffisante

\begin{proposition}
    Soit $a, n \in \mathbb{Z}$, alors

    \[ a \perp n \implies {[a]}_n \in \mathfrak{J} \]
\end{proposition}

\begin{proof}
    Soit $[a] \in S_n^1$, on prend $a \in [a]$ son représentant naturel $(a < n)$

    Si $a = 1$, alors $[a] \in \mathfrak{J}$

    Sinon, prenons le singleton $E_a = \{ (-n \mod a) \in S_n^1 \}$

    Il suffit de montrer que $(-n \mod a) \perp a$ pour répéter le raisonnement

    Or on a, grâce aux propriétés de $\gcd$
    \begin{align*}
        a \perp n &\implies a \perp n - a \\
        &\implies a \perp a - n \\
        &\implies a \perp (-n \mod a) \\
        &\implies (-n \pmod a) \perp a
    \end{align*}
\end{proof}

\section{Conjectures}

\begin{conjecture}
    Soit $a=(a_0, a_1, a_2)$ un éclatement d'une singularité $[a]$

    Posons $s = a_1 + a_2 + \gcd(a_0-a_1, a_0-a_2)$

    Supposons $a_0 < s$

    Alors $[a]$ est de type $\mathfrak{J}$ $\implies s = a_0 + 1$
\end{conjecture}

On constate que la réciproque n'est pas vraie, par exemple prenons $(13, 7, 4)$, on a $s = 14$, vérifiant donc $a_0 < 14$ et $a_0 + 1 = 14$, or elle n'est pas de type $\mathfrak{J}$.

\begin{conjecture}
    Soit $a=(a_0, a_1, a_2)$ un éclatement d'une singularité $[a]$

    Alors $[a]$ est de type $\mathfrak{J}$ $\implies \exists p, q: a_0 = p*a_1 + q*a_2$
\end{conjecture}

\begin{conjecture}
    (strict) Soit $[a] = ([a_1], \dots, [a_m]) \in \mathfrak{J}, m \geq 2$

    Alors $|\{ a_i \in [a] \mid a_i == 1 \} \geq m - 2$
\end{conjecture}

\begin{conjecture}
    Si $[a] \in \mathfrak{J}$, alors son représentant naturel $a \in [a]$ offre une suite d'éclatement montrant $[a] \in \mathfrak{J}$
\end{conjecture}


\end{document}
