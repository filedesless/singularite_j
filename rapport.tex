\documentclass{article}
\usepackage{graphicx} % Required for inserting images
\usepackage{amsthm}
\usepackage{amsmath}
\usepackage{amssymb}

\title{Caractérisation des singularités de type $\mathfrak{J}$}
\author{Félix Larose-Gervais}
\date{Mai 2023}

\newtheorem{proposition}{Proposition}
\newtheorem{conjecture}{Conjecture}

\begin{document}

\maketitle

\section{Introduction}

\newpage

\section{Propositions}

Soit $n \in \mathbb{N}, \sigma \in S_n$, notons
\begin{align*}
    \pi : \mathbb{N}^{n+1} & \to \mathbb{N}^{n+1}                               \\
    (a_0, a_1, \dots, a_n) & \mapsto (a_0, a_{\sigma(1)}, \dots, a_{\sigma(n)})
\end{align*}

\begin{proposition}
    Soit $a=(a_0, a_1, \dots, a_n) \in \mathbb{N}^{n+1}$, alors
    $$\mathfrak{J}(a) \iff \mathfrak{J}(\pi(a))$$
\end{proposition}

\begin{proof}
    Notons $b = \pi(a)$

    On a $E_a = \{ a^i \mid i = 1..n,\; a_i > 1 \},\; a^i = (a_i, a_1^i, \dots, a_n^i)$

    Avec $\forall j = 1..n,\; a_j^i = \begin{cases}
            -a_0 \mod a_i & \text{si } i = j \\
            a_j \mod a_i  & \text{sinon}
        \end{cases}$

    Considérons $E_b = \{b^i \mid i = 1..n,\; b_i > 1\},\; b^i = (b_i, b_1^i, \dots, b_n^i)$

    Avec $\forall j = 1..n$ \begin{align*}
        b_j^i & = \begin{cases}
                      -b_0 \mod b_i & \text{si } i = j \\
                      b_j \mod b_i  & \text{sinon}
                  \end{cases}
        = \begin{cases}
              -a_0 \mod a_{\sigma(i)}          & \text{si } \sigma(i) = \sigma(j) \\
              a_{\sigma(j)} \mod a_{\sigma(i)} & \text{sinon}
          \end{cases}
    \end{align*}

    Donc $b_j^i= a_{\sigma(j)}^{\sigma(i)}$

    Donc $b^i = (a_{\sigma(i)}, a_{\sigma(1)}^{\sigma(i)}, \dots, a_{\sigma(n)}^{\sigma(i)}) = \pi(a^{\sigma(i)})$

    Ainsi $E_b = \{ \pi(a^{\sigma(i)}) \mid i = 1..n,\; a_{\sigma(i)} > 1 \}$

    C'est-à-dire $E_b = \{ \pi(a^i) \mid i = 1..n,\; a_i > 1\}$

    Donc $\pi$ est une bijection entre $E_a$ et $E_b$

    Et elle préserve le caractère lisse

    S'il existe une suite d'éclatements montrant $\mathfrak{J}(a)$

    L'application de $\pi$ à ces éléments est une suite montrant $\mathfrak{J}(b)$

    La réciproque est vraie car $a = \pi^{-1}(\pi(a))$
\end{proof}

\newpage

\begin{proposition}
    Soit $a=(a_0, a_1, a_2)$, alors
    $$\mathfrak{J}(a) \implies a_0 \geq a_1 + a_2$$
\end{proposition}

\begin{proof}
    Supposons $a_0 < a_1 + a_2$

    Si $a_1 = a_2$, alors $\neg \mathfrak{J}(a)$

    Sinon, $a_1 \neq a_2$, supposons sans perdre de généralité que $a_1 > a_2$

    Considérons l'éclatement $a^1 = (a_1,\; -a_0 \mod a_1,\; a_2 \mod a_1) \in E_a$
    \begin{align*}
        a_1 > a_2 & \implies 2a_1 > a_1+a_2               \\
                  & \implies (-a_0 \mod a_1) = 2a_1 - a_0 \\
        a_1 > a_2 & \implies (a_2 \mod a_1) = a_2
    \end{align*}

    On a donc $a^1 = (a_1, 2a_1-a_0, a_2)$

    Puisque $a_0 < a_1 + a_2$, on a $a_1 < 2a_1 - a_0 + a_2$

    Donc $a^1$ vérifie la condition initiale, on répète le raisonnement avec $a^1$
\end{proof}

\section{Conjectures}

\begin{conjecture}
    Soit $a=(a_0, a_1, a_2)$ un éclatement d'une singularité $[a]$

    Posons $s = a_1 + a_2 + \gcd(a_0-a_1, a_0-a_2)$

    Supposons $a_0 < s$

    Alors $[a]$ est de type $\mathfrak{J}$ $\implies s = a_0 + 1$
\end{conjecture}

On constate que la réciproque n'est pas vraie, par exemple prenons $(13, 7, 4)$, on a $s = 14$, vérifiant donc $a_0 < 14$ et $a_0 + 1 = 14$, or elle n'est pas de type $\mathfrak{J}$.

\begin{conjecture}
    Soit $a=(a_0, a_1, a_2)$ un éclatement d'une singularité $[a]$

    Alors $[a]$ est de type $\mathfrak{J}$ $\implies \exists p, q: a_0 = p*a_1 + q*a_2$
\end{conjecture}


\end{document}
