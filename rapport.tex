\documentclass{article}
\usepackage{graphicx} % Required for inserting images
\usepackage[utf8]{inputenc}
\usepackage{amsthm}
\usepackage{amsmath}
\usepackage{amssymb}

\title{Caractérisation des singularités de type $\mathfrak{J}$}
\author{Félix Larose-Gervais}
\date{Mai 2023}


\newtheorem{definition}{Définition}
\newtheorem{proposition}{Proposition}
\newtheorem{conjecture}{Conjecture}
\newtheorem{exemple}{Exemple}

\begin{document}

\maketitle

\newpage

\tableofcontents

\newpage

\section{Introduction}

\subsection{Définitions}

Soit $r, n \in \mathbb{N}$, notons $\mathbb{Z}_r$ l'anneau $\mathbb{Z}/r\mathbb{Z}$ et $\mathbb{Z}_r^\times$ son groupe d'inversibles

\begin{definition}
    Une \textbf{singularité} est un $[a] = ([a_1], \dots, [a_n]) \in (\mathbb{Z}_r^\times)^n$, on appelle \begin{itemize}
        \item $r$ la \textbf{racine} de la singularité
        \item $[a_1], \dots, [a_n]$ les \textbf{poids} de la singularité
    \end{itemize}
\end{definition}

\begin{definition}
    Un \textbf{éclatement} $a \in \mathbb{Z}^n$ d'une singularité $[a]$ (noté $a \in [a]$) est un choix de 
    représentant $a = (a_1, \dots, a_n)$ tel que
    \begin{align*}
         \gcd(a_i, a_j) = 1 \tag{$\forall i \neq j$}
    \end{align*}
    On note $E_a$ l'ensemble des singularités associées à l'éclatement $a$ comme suit:
    \begin{align*}
        E_a = \{ ([a_1^i], \dots, [a_n^i]) \mid \forall i = 1..n,\; a_i > 1, [a_1^i], \dots, [a_n^i] \in \mathbb{Z}_{a_i}^\times \} \\
        \forall j = 1..n : [a^i_j] \equiv \begin{cases}
            -r & \text{si $i = j$} \\
            a_j & \text{sinon}
        \end{cases} \pmod{a_i}
    \end{align*}
\end{definition}

\begin{definition}
    Un éclatement $a \in [a]$ est dit \textbf{lisse} si $E_a = \emptyset$
\end{definition}

\begin{definition}
    La singularité $[a]$ est dite de \textbf{type $\mathfrak{J}$} (noté $[a] \in \mathfrak{J}$) ssi
    \begin{align*}
        \exists a \in [a] : \forall [a^i] \in E_a : [a^i] \in \mathfrak{J}
    \end{align*}
\end{definition}
    
\subsection{Résultats connus}
Résultats utiles, dûs à Habib Jaber.

\begin{proposition}
    Soit $a_1, a_2 \in \mathbb{Z},\; \gcd(a_1, a_2) = 1$, alors
    \[ [(a_1 + a_2, a_1, a_2)] \in \mathfrak{J} \]
\end{proposition}

\begin{exemple}
    $\gcd(2, 1) = 1 \implies [(3, 2, 1)] \in \mathfrak{J}$
\end{exemple}

\begin{proposition}
    \[ 
    [(a_0, a_1, a_2)] \in \mathfrak{J} \iff 
    \forall k \in \mathbb{Z}: [(a_0 + k(a_1a_2), a_1, a_2)] \in \mathfrak{J} 
    \]
\end{proposition}

\begin{exemple}
    $[(3, 2, 1)] \in \mathfrak{J} \implies [(5, 2, 1)] \in \mathfrak{J},\; [(7, 2, 1)] \in \mathfrak{J}, \dots$
\end{exemple}

\newpage

\section{Propositions}

Soit $\sigma \in S_n$, notons la permutation $\pi$
\begin{align*}
    \pi : \mathbb{Z}^n_r & \to \mathbb{Z}^n_r \\
    ([a_1], \dots, [a_n]) & \mapsto ([a_{\sigma(1)}], \dots, [a_{\sigma(n)}])
\end{align*}

\begin{proposition}
    L'ordre des poids d'une singularité n'affecte pas le type $\mathfrak{J}$

    Soit $[a]=([a_1], \dots, [a_n]) \in (\mathbb{Z}_r^\times)^n$, alors
    \[ [a] \in \mathfrak{J} \implies \pi([a]) \in \mathfrak{J} \]
\end{proposition}

\begin{proof}
    Notons $[b] = \pi([a])$

    On a, pour $a$ l'éclatement trivial de $[a]$ 
    \begin{align*}
        E_a & = \{ ([a^i_1], \dots, [a^i_n]) \mid \forall i = 1..n,\; a_i > 1,\; [a_1^i], \dots, [a_n^i] \in \mathbb{Z}_{a_i}^\times \} \\
        [a_j^i] &\equiv \begin{cases}
                -r &\text{si } i = j \\
                a_j &\text{sinon}
            \end{cases} \pmod{a_i} \quad \forall j = 1..n
    \end{align*}

    Considérons, pour $b$ l'éclatement trivial de $[b]$
    \begin{align*}
        E_b & = \{ ([b^i_1], \dots, [b^i_n]) \mid \forall i = 1..n,\; b_i > 1,\; [b_1^i], \dots, [b_n^i] \in \mathbb{Z}_{b_i}^\times\} \\
        [b_j^i] & \equiv \begin{cases}
                        -r &\text{si } i = j \\
                        b_j &\text{sinon}
                    \end{cases} \pmod{b_i} \\
            & \equiv \begin{cases}
                -r &\text{si } \sigma(i) = \sigma(j) \\
                a_{\sigma(j)} &\text{sinon}
            \end{cases} \pmod {a_{\sigma(i)}} \quad \forall j = 1..n
    \end{align*}

    Donc $[b_j^i] = [a_{\sigma(j)}^{\sigma(i)}]$

    Donc $[b^i] = ([a_{\sigma(1)}^{\sigma(i)}], \dots, [a_{\sigma(n)}^{\sigma(i)}]) = \pi([a^{\sigma(i)}])$

    Ainsi $E_b = \{ \pi([a^{\sigma(i)}]) \mid \forall i = 1..n,\; a_{\sigma(i)} > 1 \} = \{ \pi(a^i) \mid \forall i = 1..n,\; a_i > 1\}$

    Donc $\pi$ est une bijection entre $E_a$ et $E_b$

    Et elle préserve le caractère lisse

    S'il existe une suite d'éclatements montrant $[a] \in \mathfrak{J}$

    L'application de $\pi$ à ces éléments est une suite montrant $[b] \in \mathfrak{J}$
\end{proof}

\begin{exemple}
    Sachant $(5, 3, 1) \in \mathfrak{J}$, on en déduit $(5, 1, 3) \in \mathfrak{J}$
\end{exemple}

\newpage

\begin{proposition}
    Soit $a=(a_0, a_1, a_2)$, alors
    \[ a \in \mathfrak{J} \implies a_0 \geq a_1 + a_2 \]
\end{proposition}

\begin{proof}
    Supposons $a_0 < a_1 + a_2$

    Si $a_1 = a_2$, alors $\neg \mathfrak{J}(a)$

    Sinon, $a_1 \neq a_2$, supposons sans perdre de généralité que $a_1 > a_2$

    Considérons l'éclatement $a^1 = (a_1,\; -a_0 \mod a_1,\; a_2 \mod a_1) \in E_a$
    \begin{align*}
        a_1 > a_2 & \implies 2a_1 > a_1+a_2               \\
                  & \implies (-a_0 \mod a_1) = 2a_1 - a_0 \\
        a_1 > a_2 & \implies (a_2 \mod a_1) = a_2
    \end{align*}

    On a donc $a^1 = (a_1, 2a_1-a_0, a_2)$

    Puisque $a_0 < a_1 + a_2$, on a $a_1 < 2a_1 - a_0 + a_2$

    Donc $a^1$ vérifie la condition initiale, on répète le raisonnement avec $a^1$
\end{proof}

\begin{exemple}
    $(5, 4, 3) \not \in \mathfrak{J}$ car $5 < 7$
\end{exemple}

\section{Conjectures}

\begin{conjecture}
    Soit $a=(a_0, a_1, a_2)$ un éclatement d'une singularité $[a]$

    Posons $s = a_1 + a_2 + \gcd(a_0-a_1, a_0-a_2)$

    Supposons $a_0 < s$

    Alors $[a]$ est de type $\mathfrak{J}$ $\implies s = a_0 + 1$
\end{conjecture}

On constate que la réciproque n'est pas vraie, par exemple prenons $(13, 7, 4)$, on a $s = 14$, vérifiant donc $a_0 < 14$ et $a_0 + 1 = 14$, or elle n'est pas de type $\mathfrak{J}$.

\begin{conjecture}
    Soit $a=(a_0, a_1, a_2)$ un éclatement d'une singularité $[a]$

    Alors $[a]$ est de type $\mathfrak{J}$ $\implies \exists p, q: a_0 = p*a_1 + q*a_2$
\end{conjecture}


\end{document}
